%%%%%%%%%%%%%%%%%%%%%%%%%%%%%%%%%%%%%%%%%%%%%%%%%%%%%%%%%%%
\section{Multi-Actuator Design Constraints}
\label{sec:analysis}
%%%%%%%%%%%%%%%%%%%%%%%%%%%%%%%%%%%%%%%%%%%%%%%%%%%%%%%%%%%
The preceding section demonstrated how closed-loop control can independently control multiple rotors.  Given this capability, there are constraints that must be respected when designing an MRI powered and controlled multi-actuator system.  These involve (1) arranging the rotors to minimize interaction forces, (2) MRI imaging-based tracking of each rotor, (3) geometrically arranging the rotor axes to maximize torque, and (4) calculating the stall torque as a function of the number of actuators. Each of these is described below.

%Using multiple rotors simultaneously in the same MR scanner introduces challenges not seen with only one rotor.  This section analyzes the magnetic interaction between the ferrous spheres, interference when imaging, and questions about the optimal placement of the rotors.


\subsection{Actuator Interaction Forces}\label{subsec:minimumseparation}
Any ferrous material placed in the magnetic field of an MR scanner becomes a strong magnetic dipole.  With multiple MR-powered motors, these dipoles exert forces on each other.  Dipole forces overpower MRI gradient forces if rotors are closer than a threshold distance.

The magnetic field at position $\mathbf{r}_2$ generated by a spherical magnet at position $\mathbf{r}_1$ with magnetization  $\mathbf{m}_1$ is  \cite{Schill2003} %\cite{thomaszewski2008magnets}
\begin{align}
\label{eq:dipoleMagField}
 \mathbf{B}_{\mathbf{r}_1}(\mathbf{r_2}) = \frac{\mu_0}{4 \pi}\frac{3 \mathbf{n}_{12}(\mathbf{n}_{12} \cdot \mathbf{m}_1) - \mathbf{m}_1}
 {|\mathbf{r}_2-\mathbf{r}_1|^3},
\end{align}
with  $\mathbf{n}_{12} = (\mathbf{r}_2-\mathbf{r}_1)/|\mathbf{r}_2-\mathbf{r}_1|$. This is the \emph{magnetic field of a dipole}.
 The force applied to a dipole at $\mathbf{r}_1$ with magnetic moment $\mathbf{m}_1$ by another dipole at $\mathbf{r}_2$ with magnetic moment $\mathbf{m}_2$ is approximated by
\begin{align}
\mathbf{F}_{12} \approx \frac{3\mu_0}{4 \pi} \frac{1}{|\mathbf{r}_2 - \mathbf{r}_1 |^4}
\left[5 \mathbf{n}_{12}\Big(\left(\mathbf{m}_1 \cdot \mathbf{n}_{12} \right)   \left(\mathbf{m}_2 \cdot \mathbf{n}_{12} \right) \Big) \right. \nonumber \\
\left.
-  \mathbf{n}_{12} \left(\mathbf{m}_2 \cdot \mathbf{m}_1 \right)
-  \mathbf{m}_{1} \left(\mathbf{m}_2 \cdot \mathbf{n}_{12} \right)  -  \mathbf{m}_{2} \left(\mathbf{m}_1 \cdot \mathbf{n}_{12}\right)   \right].\nonumber
\label{eq:dipoleForce}
\end{align}

 \begin{figure}
 \centering
\begin{overpic}[height = 0.47\columnwidth]{MagneticDipoleField3mmRast}
\tiny
\put(28,29){0}
\put(28,73){0}
\put(89,73){0}
\put(89,29){0}
\put(27,51){-0.1$g_{M}$}
\put(52,17){0.1$g_{M}$}
\put(56,38){$g_{M}$}
\put(64.5,52){-$g_{M}$}
\small
\put(35,-8){$r_{sphere}$ = 3mm}
\end{overpic}
\begin{overpic}[height = 0.47\columnwidth]{MagneticDipoleField6mmRast}
\tiny
\put(20,29){0}
\put(20,79){0}
\put(89,79){0}
\put(89,30){0}
\put(15,70){-0.1$g_{M}$}
\put(25,15){0.1$g_{M}$}
\put(52,38){$g_{M}$}
\put(68.5,55){-$g_{M}$}
\small
\put(30,-8){$r_{sphere}$ = 6mm}
 \end{overpic}
 \vspace{-1em}
%\begin{overpic}[height = 0.47\columnwidth]{MagneticDipoleField4mmRast}
%\tiny
%\put(23,29){0}
%\put(15,55){-0.1$g_{M}$}
%\put(52,17){0.1$g_{M}$}
%\put(52,38){$g_{M}$}
%\put(64.5,55){-$g_{M}$} \end{overpic}
\caption{\label{fig:MagneticDipoleField3mmRast}One ferrous sphere in a 3T magnetic field exerts a  force $\mathbf{F}$ on an identical sphere.   The contour lines show $\mathbf{F}\cdot \mathbf{n}_{12}$, the force component radially outward from the sphere at $(0,0)$ compared to the maximum force provided by the gradient coils $g_{M}$.  This force is attractive (red) along the $z$-axis and repulsive (blue) perpendicular to $z$. The magnetic field is symmetric about the $z$-axis.  If two spheres move within the dark red region, they cannot be separated using the gradient field. }
\vspace{-1em}
\end{figure}
Figure \ref{fig:MagneticDipoleField3mmRast} shows contour plots for the magnetic force exerted by two identical spheres on each other.  The contour lines are drawn at $\mathbf{F}_{12}\cdot \mathbf{n}_{12} = g_{M}\cdot \{-1,-\frac{1}{10},0,\frac{1}{10},1\}$.  Rotors with spheres closer than the $g_{M}$ contour lines will become stuck because they experience a force greater than what the gradient can exert.  The maximum force is along the $z$-axis, and the critical distance when the attractive force becomes greater than the maximum gradient force is $\sqrt[4]{\frac{2  \mathbf{M}_s \mu_0 r^3_{sphere} }{g_{M}}}. $  This interaction decays quickly and at distance $\approx 5.4 r_{sphere}^{3/4}$ is  10\% of the maximum gradient. The required distance, $d$, to ensure dipole-dipole forces are less than some $percentage$ of the maximum gradient is given by
%but rather the distance at which you can ignore the interaction - e.g., it only reduces max torque by 10\%. 
\begin{equation}
d \ge \sqrt[4]{\frac{ 2 \frac{100}{percentage} M_{sz} \mu_0 r^3_{sphere} }{g_{M}}}.
\label{eq:dipoledipolePercentGrad}
\end{equation}




\subsection{Simultaneous Tracking of Multiple Rotors}\label{subsec:FeedbackSensing}
An MR scanner can provide both power and feedback sensing for closed-loop motor control. Though ferrous spheres cannot be imaged with an MR scanner, a sphere can be used to selectively discriminate the resonance frequency of a fiducial marker placed a set distance from the sphere.

This method was used in \cite{Vartholomeos2013} to track a single rotor.  Attaching the fiducial marker to the same axle as the ferrous sphere, with an axial offset as shown in Fig~\ref{fig:Schematic}, allows use of a single, configuration-independent RF-frequency to image the marker.
To image a marker, the offset resonance frequency of the excitation Radio Frequency (RF) pulse is
\begin{equation}
\Delta f(d) = \frac{\gamma B_z(d)}{2\pi}
\label{eq:resonanceFreq}
\end{equation}
Here, $\Delta f$ [Hz] is the RF offset, $\frac{\gamma}{2\pi}$ is the gyromagnetic ratio where $\gamma$ is 42.57MHz/T, and $B_z(d)$ is the magnitude in Tesla of the magnetic field induced by the ferrous sphere at distance $d$ from the marker.
Real-time 2D tracking of the rotor is accomplished by first acquiring two orthogonal projections, and then using a peak detection algorithm to locate the marker.

Localizing several rotors is difficult because their projections can overlap.  One method to avoid overlapping signals uses unique distances between marker and ferrous sphere on each rotor. 
 By appropriate choice of offset resonance frequency and its bandwidth, only one rotor at a time is visible on any acquired projection. However, this method requires an additional tracking sequence for each rotor.   

A faster alternative is to design the rotors and projections so the paths of the markers do not intersect in any projection. In this way, $n$ rotors can be simultaneously tracked with a single acquisition sequence, followed by detecting $n$ non-intersecting peaks on each projection. This approach is illustrated in Fig.~\ref{fig:MRItrackingSequence}, showing three orthogonal projections for tracking three orthogonal rotors.  This tracking sequence requires 25ms, enabling real-time positioning of the rotors.
For this method to work, each marker must be disjoint in at least two non-parallel projections, and these projections must not be parallel with the axis of rotation. Reconstructed rotor positions from an experiment with three parallel rotors are depicted in Fig.~\ref{fig:MarkerRead3rotors}. 


 \begin{figure}
 \centering
\begin{overpic}[width = \columnwidth]{MRItrackingSequence.pdf}\end{overpic}
\vspace{-2em}
\caption{\label{fig:MRItrackingSequence}MRI Fast Spin Echo sequence for tracking three orthogonal rotors.}
\vspace{-1em}
\end{figure}


 \begin{figure}
 \centering
  \begin{minipage}{0.7\linewidth}
\begin{overpic}[width = .925\columnwidth]{MarkerRead3rotorsCrop.pdf}
\end{overpic}\end{minipage}\hspace{-1em}
 \begin{minipage}{0.3\linewidth}
 \begin{overpic}[width = \columnwidth]{MRI3rotorsFar.jpg}\end{overpic} \vspace{-1.em}\\
 \vspace{-.5em}
\footnotesize MRI \\ \vspace{.05em}
\footnotesize~~~~~~~~~~~~detail view\\ \vspace{-.85em}
\begin{overpic}[width = \columnwidth]{MRI3rotorsClose.jpg}\end{overpic}
\end{minipage}
\caption{\label{fig:MarkerRead3rotors}Simultaneous tracking of three rotors with two line scans.}
\vspace{-2em}
\end{figure}

\subsection{Optimal Geometric Arrangement of $n$ Rotors}\label{subsec:optimalrotorplacement}
Controllability depends on both the geometric arrangement and the physical properties of the rotors, e.g.~$\theta_i(0), r_i, v_i$. Because the physical properties are chosen to meet torque requirements, this section focuses on maximizing controllability via arranging the rotor axes of rotation.

Controller \eqref{eq:velocityControlPolicyInv} exploits inhomogeneity between motor rotors.   %NOTE: "well-spaced" is a technical math term
Inhomogeneity is maximized geometrically when the axes' orientations are \emph{well spaced}, that is all axes are as far from being parallel as possible. Section \ref{subsec:TorqueFuncN} shows that well-spaced axes maximize output torque. Fortuitously, if the rotors are arranged on the surface of a hemisphere, well-spaced axes are also maximally separated.  This minimizes the dipole-dipole forces described in Sec. \ref{subsec:minimumseparation}.  
%Because the rotors will be used in the uniform gradient field of an MRI, their physical location is not important, as long rotor spacing is sufficient to avoid the dipole-dipole forces described in Sec. \ref{subsec:minimumseparation}.  
%Instead, it is important to generate axes of rotation, the unit vectors ${\bf{a}}_1,{\bf{a}}_2,\ldots,{\bf{a}}_n$, ${\bf{a}}_i\in \R^{3}$, that are non-parallel.  

This problem is a variant of the Thomson problem \cite{Thomson1904} which determines the minimum energy configuration for $n$ electrons confined to the surface of a sphere.  In this variant, to each of the $n$ electrons located at ${\bf{a}}_i\in \R^{3}, ||{\bf{a}}_i||_2=1$, an additional electron at $-{\bf{a}}_i$ is bound, and the system is solved to minimize the total energy.  As in the original Thomson problem, minimal energy configurations can be rigorously identified in only a handful of cases. Instead, as in \cite{Peng2012}, this paper uses numerical optimization methods to find locally optimal solutions.

The optimization problem is
\begin{align}
\underset{{\bf{a}}_1,{\bf{a}}_2,\ldots,{\bf{a}}_n}{\text{minimize}}  & \sum_{i\ne j} \frac{1}{\norm{{\bf{a}}_i-{\bf{a}}_j}_2^2 }  + \frac{1}{\norm{{\bf{a}}_i+{\bf{a}}_j}_2^2 } \nonumber\\
\text{subject to } & {\bf{a}}_i\in \R^3, \norm{{\bf{a}}_i}_2 = 1,\, 1\le i\le n.
\label{eq:minimizationAxesCartesian}
\end{align}
Both the objective function and the constraints are nonconvex, but \eqref{eq:minimizationAxesCartesian} can be reformulated as an unconstrained problem by changing to a spherical coordinate system parameterized by azimuth $\lambda$ and elevation $\phi$:
\begin{align*}    % lat is commonly \phi;  longitude is \lambda
x = \cos(\phi)\sin(\lambda), \quad
y =  \cos(\phi)\cos(\lambda), \quad
z =  \sin(\phi).
\end{align*}
The original problem had $3n$ variables and $n$ constraints.  Using spherical coordinates results in $2n$ variables and no constraints. Using the shorthand $c_\theta = \cos(\theta), s_\theta = \sin(\theta)$, 
 the objective function \eqref{eq:minimizationAxesCartesian}  can be recomputed as
\begin{align}
f = \sum_{j=1}^{n} \sum_{i=1}^{j} & \frac{1}{2\left(1- c_{\phi_i}c_{\phi_j}c_{\lambda_i-\lambda_j} - s_{\phi_i}s_{\phi_j} \right) }+\nonumber\\
						& \frac{1}{2\left(1+ c_{\phi_i}c_{\phi_j}c_{\lambda_i-\lambda_j} + s_{\phi_i}s_{\phi_j} \right) },
%f = \sum_{j=1}^{n} \sum_{i=1}^{j} & \frac{1}{2\left(1- \cos(\phi_i)\cos(\phi_j)\cos(\lambda_i-\lambda_j) - \sin(\phi_i)\sin(\phi_j) \right) }\\
%						+& \frac{1}{2\left(1+ \cos(\phi_i)\cos(\phi_j)\cos(\lambda_i-\lambda_j) + \sin(\phi_i)\sin(\phi_j)\right)  }
\label{eq:objectiveForSpherical}
\end{align}
and the gradient calculated as
\begin{align}
\frac{\partial f}{\partial \phi_i} =  & \frac{c_{\phi_i}c_{\lambda_i-\lambda_j}s_{\phi_i} - c_{\phi_i}s_{\phi_i}}{2\left(1- c_{\phi_i}c_{\phi_j}c_{\lambda_i-\lambda_j} - s_{\phi_i}s_{\phi_j} \right)^2 }+\nonumber\\
						& \frac{c_{\phi_i}c_{\lambda_i-\lambda_j}s_{\phi_i} - c_{\phi_i}s_{\phi_i}}{2\left(1+ c_{\phi_i}c_{\phi_j}c_{\lambda_i-\lambda_j} + s_{\phi_i}s_{\phi_j} \right)^2 }\nonumber \\
\frac{\partial f}{\partial \lambda_i} =  & \frac{c_{\phi_i}c_{\phi_j}s_{\lambda_i-\lambda_j} }{2\left(1- c_{\phi_i}c_{\phi_j}c_{\lambda_i-\lambda_j} - s_{\phi_i}s_{\lambda_j} \right)^2 }+\nonumber\\
						& \frac{c_{\phi_i}c_{\phi_j}s_{\lambda_i-\lambda_j} }{2\left(1+ c_{\phi_i}c_{\phi_j}c_{\lambda_i-\lambda_j} + s_{\phi_i}s_{\lambda_j} \right)^2 }.
\label{eq:gradientForSpherical}
\end{align}




 \begin{figure}
\begin{overpic}[width = \columnwidth]{OptimalAxisFigCrop}
\put(8,-4){$n$=4}
\put(34,-4){$n$=5}
\put(58,-4){$n$=6}
\put(82,-4){$n$=24}
\end{overpic}
\vspace{-1em}
\caption{\label{fig:OptimalAxisFig}Numerical optimization of rotor axis spacing for different numbers of axes, $n$.  The rotation axes are defined by lines from each blue vertex through the origin to the corresponding red vertex.  Arrangements from left to right: cube, pentagonal antiprism, icosahedron, and irregular.
}
\vspace{-2em}
\end{figure}

\href{http://www.mathworks.com/matlabcentral/fileexchange/44515}{{\sc Matlab} code implementing gradient descent on} \eqref{eq:objectiveForSpherical} using \eqref{eq:gradientForSpherical} \href{http://www.mathworks.com/matlabcentral/fileexchange/44515}{to find locally optimal solutions is available at} \cite{Becker2013j}. Example output is shown in Fig.~\ref{fig:OptimalAxisFig}. 
%As in the Thomson problem, minimal energy conditions have not been identified for all $n$. For $n$=2 and 3 the axes must be orthogonal. 


\subsection{Stall Torque versus Number of Actuators}\label{subsec:TorqueFuncN}
Two effects must be considered when computing stall torque. The first is related to the directionality of the maximum gradient that can be produced by a scanner. The magnetic gradients in the three coordinate directions are produced by three separate coils and amplifiers. The maximum gradient that can be applied in each direction depends on the maximum current that each coil is designed to handle. The practical implication is that the maximum gradient that can be generated is not directed along one of the three principal coordinate axes, but occurs off-axis when the three gradient coils are all producing their maximum values. The result is that, for any given rotor axis, the maximum torque varies cyclically with rotation angle. 

The second effect arises because control effort must be divided among many rotors. %Independent control requires inhomogeneity between the rotors.  Inhomogeneity from orienting the rotors along different axis of rotation enables independent control, but the torque gain is sublinear in the number of rotors because the rotors are not aligned.  
This section analyzes the average torque produced with 1, 2, 3, or $n$ rotors and how geometrically arranging the rotor axes modifies this torque.

%\todo{preceding paragraph is terrible}

 \paragraph{Single rotor}
 Consider one rotor aligned along the MRI $x$-axis. The state is $[\theta_x, \dot{\theta}_x]^\intercal$.  Actuating the $F_y$ and $F_z$ gradient fields  imparts a torque on this rotor.  This analysis compares the \emph{stopped torque}, the torque applied to a stationary rotor,  assuming without loss of generality that the velocity error is +1.  After setting the maximum gradient, rotor length, saturation magnetization, and inertia to 1, the stopped torque under control law \eqref{eq:velocityControlPolicy}  as a function of $\theta_x$ is
 \begin{align}
%T_x =  -\sgn\left(-\cos(\theta_x)\right) \cos(\theta_x) + 
%  \sgn\left(\sin(\theta_x)\right) \sin(\theta_x)
 \tau_x =  \sgn\left(c_{\theta_x}\right) c_{\theta_x} + 
  \sgn\left(s_{\theta_x}\right) s_{\theta_x}.
  \label{eq:torque1rotor100}
\end{align} 
 Integrating \eqref{eq:torque1rotor100} over $\theta_x$ produces an average torque of $\frac{4}{\pi} \approx 1.27$ Nm.  Equation \eqref{eq:torque1rotor100} is plotted in Fig.~\ref{fig:RotorTorque1Rotor}.  The three MRI gradients can be independently maximized. This can be exploited by picking a rotor axis such that each gradient contributes torque.   A rotor spinning around the axis [1,1,0] or [1,1,1] generates larger average torques than [1,0,0].
 
 \begin{figure}
 \centering
\begin{overpic}[width =.8 \columnwidth]{RotorTorque1Rotor}
\put(38,15){$x$}
\put(42,23){$y$}
\put(20,40){$z$}
\end{overpic}
\vspace{-1em}
\caption{
\label{fig:RotorTorque1Rotor}
With one rotor that rotates about a magnetic-field axis, the average stopped torque is $\frac{4}{\pi}$, with minimum 1 and maximum $\sqrt{2}$. Rotating about the vector $[1,1,0]$ or $[1,1,1]$ generates slightly larger average torques.
\vspace{-2em}
}
\end{figure}
 
  \paragraph{Two rotors}  For multiple rotors, the average torque is calculated by dividing the sum stopped torque of all the rotors by the number of rotors.  With two orthogonal rotors oriented along the MRI $x$ and $y$ axes, each rotor torque averages  $\frac{2 + \pi}{\pi^2} \approx 1.04$Nm.  %The minimum torque is 0 and the maximum is $2\sqrt{2}\approx2.83$, as shown in Fig.~\ref{fig:RotorTorque2Rotors}.  
%   \begin{figure}
%\begin{overpic}[height = 0.58\columnwidth]{RotorTorque2Rotors}\end{overpic}
%\begin{overpic}[height = 0.58\columnwidth]{RotorTorque2RotorsColorbar}\end{overpic}
%\caption{\label{fig:RotorTorque2Rotors}The average total torque with two orthogonal rotors spinning about $[1,0,0]$ and $[0,1,0]$ is $\approx$2.08. The individual torques average $\approx$1.04.
%}
%\end{figure}

\paragraph{Three rotors}
With three orthogonal rotors oriented along the MRI $x,y,$ and $z$ axes,  the torques produced are
   \begin{align}  %(\theta_x,\theta_y,\theta_z)
    \label{eq:torque3rotor}
  \tau_x &= \sgn( c_{\theta_x} -  s_{\theta_y}) c_{\theta_x}    + \sgn(s_{\theta_x} - c_{\theta_z} ) s_{\theta_x} \nonumber\\
  \tau_y  &=  \sgn( c_{\theta_y} -  s_{\theta_z}) c_{\theta_y} +  \sgn(s_{\theta_y} - c_{\theta_x} ) s_{\theta_y}\\  
  \tau_z  &= \sgn( c_{\theta_z} -  s_{\theta_x}) c_{\theta_z}  +   \sgn( s_{\theta_z} - c_{\theta_y} ) s_{\theta_z}\nonumber \\
 \bar{\tau}_3  &= \frac{1}{3}\frac{1}{(2\pi)^2} \int_0^{2\pi} \!\! \int_0^{2\pi} \!\! \int_0^{2\pi}   \tau_x +  \tau_y +  \tau_z  \dd\theta_x \dd\theta_y \dd\theta_z.\nonumber
\end{align} 
 Each rotor averages a stopped torque of $ \bar{\tau}_3=\frac{8}{\pi^2}\approx0.81$. 

\paragraph{$n$ rotors}
With $n$ rotors, the average stopped torque is calculated by integrating over each angle $\theta_i$ and dividing by $n$:
   \begin{align}  
      \bar{\tau}_n = &\frac{1}{n} \frac{1}{(2\pi)^n} \!\!  \underbrace{\int_0^{2\pi} \ldots \int_0^{2\pi}}_{n} \sum_{i=1}^{n} \tau_i   \dd\theta_1 \ldots \dd\theta_n
  \label{eq:torquenrotor}
\end{align} 

 \begin{figure}
 \centering
\begin{overpic}[width = 0.49\columnwidth]{AveStoppedTorqueSum.pdf}\end{overpic}
\begin{overpic}[width = 0.49\columnwidth]{AveStoppedTorqueInd.pdf}\end{overpic}
\vspace{-2em}
\caption{\label{fig:AveStoppedTorqueSum}Average stopped torque as a function of the number of rotors $n$ for three different axes placement strategies.  The sum torque increases sublinearly with $n$.  The scale is normalized so 1 is the maximum torque a single gradient could impart on one rotor.  Mean and $\pm$one standard deviation are plotted. The \emph{optimized} placement strategy has the highest average torque.
}
\vspace{-2em}
\end{figure}

This integral is difficult to evaluate, so Monte Carlo simulations are used to estimate the integral.  Every data point in Fig.~\ref{fig:AveStoppedTorqueSum}  is the result of $10^6$ simulations.  Three methods for orienting rotors are compared:  
\begin{itemize}
\item\emph{Optimized:} using the numerical optimization from Section \ref{subsec:optimalrotorplacement} generates the largest average stopped torque.
\item \emph{Random:}  in spherical coordinates,  the azimuth and orientation of the rotor axis are set uniformly randomly in [0,1].  This setup produces lower average torque and erratic variance values.
\item \emph{All z-axis:} sets all rotor axis to [0,0,1].  This arrangement has no inhomogeneity.  However, the system is controllable as long as the rotors have different initial orientations: $\theta_i(0) \neq \theta_j(0)\,\, \forall i,j \in [1,n]$.  This method results in the lowest average torque.
\end{itemize}



