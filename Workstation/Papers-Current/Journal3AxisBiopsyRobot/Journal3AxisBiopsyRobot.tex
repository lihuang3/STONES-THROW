% to submit:
% DUE DATE: 


\documentclass[letterpaper, 10 pt]{IEEEtran}
\IEEEoverridecommandlockouts
\usepackage{calc}
\usepackage{url}
\usepackage[hidelinks]{hyperref}
\usepackage{graphicx}
\usepackage[cmex10]{amsmath}
\usepackage{amssymb}
\usepackage{rotating}

\usepackage{nicefrac}
\usepackage{cite}
\usepackage[caption=false,font=footnotesize]{subfig}
\usepackage[usenames, dvipsnames]{color}
\usepackage{colortbl}
\usepackage{overpic}
\graphicspath{{./pictures/pdf/},{./pictures/ps/},{./pictures/png/},{./pictures/jpg/}}
\usepackage{breqn} %for breaking equations automatically
\usepackage[ruled]{algorithm}
\usepackage{algpseudocode}
\usepackage{multirow}
\usepackage{soul}
\usepackage{bm}   % boldface math type


\newcommand{\todo}[1]{\textcolor{red}{#1}}
\newcommand{\todoRemy}[1]{\textcolor{blue}{#1}}
\newcommand{\todoAaron}[1]{\textcolor{ForestGreen}{#1}}


%% ABBREVIATIONS
\newcommand{\qstart}{q_{\text{start}}}
\newcommand{\qgoal}{q_{\text{goal}}}
\newcommand{\pstart}{p_{\text{start}}}
\newcommand{\pgoal}{p_{\text{goal}}}
\newcommand{\xstart}{x_{\text{start}}}
\newcommand{\xgoal}{x_{\text{goal}}}
\newcommand{\ystart}{y_{\text{start}}}
\newcommand{\ygoal}{y_{\text{goal}}}
\newcommand{\gammastart}{\gamma_{\text{start}}}
\newcommand{\gammagoal}{\gamma_{\text{goal}}}
\providecommand{\proc}[1]{\textsc{#1}}


\newcommand{\ARLfull}{Aero\-space Ro\-bot\-ics La\-bora\-tory }
\newcommand{\ARL}{\textsc{arl}}
\newcommand{\JPL}{\textsc{jpl}}
\newcommand{\PRM}{\textsc{prm}}
\newcommand{\CM}{\textsc{cm}}
\newcommand{\SVM}{\textsc{svm}}
\newcommand{\NN}{\textsc{nn}}
\newcommand{\prm}{\textsc{prm}}
\newcommand{\lemur}{\textsc{lemur}}
\newcommand{\Lemur}{\textsc{Lemur}}
\newcommand{\LP}{\textsc{lp}} 
\newcommand{\SOCP}{\textsc{socp}}
\newcommand{\SDP}{\textsc{sdp}}
\newcommand{\NP}{\textsc{np}}
\newcommand{\SAT}{\textsc{sat}}
\newcommand{\LMI}{\textsc{lmi}}
\newcommand{\hrp}{\textsc{hrp\nobreakdash-2}}
\newcommand{\DOF}{\textsc{dof}}
\newcommand{\UIUC}{\textsc{uiuc}}
%% MACROS


\providecommand{\abs}[1]{\left\lvert#1\right\rvert}
\providecommand{\norm}[1]{\left\lVert#1\right\rVert}
\providecommand{\normn}[2]{\left\lVert#1\right\rVert_#2}
\providecommand{\dualnorm}[1]{\norm{#1}_\ast}
\providecommand{\dualnormn}[2]{\norm{#1}_{#2\ast}}
\providecommand{\set}[1]{\lbrace\,#1\,\rbrace}
\providecommand{\cset}[2]{\lbrace\,{#1}\nobreak\mid\nobreak{#2}\,\rbrace}
\providecommand{\lscal}{<}
\providecommand{\gscal}{>}
\providecommand{\lvect}{\prec}
\providecommand{\gvect}{\succ}
\providecommand{\leqscal}{\leq}
\providecommand{\geqscal}{\geq}
\providecommand{\leqvect}{\preceq}
\providecommand{\geqvect}{\succeq}
\providecommand{\onevect}{\mathbf{1}}
\providecommand{\zerovect}{\mathbf{0}}
\providecommand{\field}[1]{\mathbb{#1}}
\providecommand{\C}{\field{C}}
\providecommand{\R}{\field{R}}
\newcommand{\Cspace}{\mathcal{Q}}
\newcommand{\Uspace}{\mathcal{U}}
\providecommand{\Fspace}{\Cspace_\text{free}}
\providecommand{\Hcal}{$\mathcal{H}$}
\providecommand{\Vcal}{$\mathcal{V}$}
\DeclareMathOperator{\conv}{conv}
\DeclareMathOperator{\cone}{cone}
\DeclareMathOperator{\homog}{homog}
\DeclareMathOperator{\domain}{dom}
\DeclareMathOperator{\range}{range}
\DeclareMathOperator{\sign}{sgn}
\DeclareMathOperator{\sgn}{sgn}
\providecommand{\polar}{\triangle}
\providecommand{\ainner}{\underline{a}}
\providecommand{\aouter}{\overline{a}}
\providecommand{\binner}{\underline{b}}
\providecommand{\bouter}{\overline{b}}
\newcommand{\D}{\nobreakdash-\textsc{d}}
%\newcommand{\Fspace}{\mathcal{F}}
\providecommand{\Fspace}{\Cspace_\text{free}}
\providecommand{\free}{\text{\{}\mathsf{free}\text{\}}}
\providecommand{\iff}{\Leftrightarrow}
\providecommand{\subinner}[1]{#1_{\text{inner}}}
\providecommand{\subouter}[1]{#1_{\text{outer}}}
\providecommand{\Ppoly}{\mathcal{X}}
\providecommand{\Pproj}{\mathcal{Y}}
\providecommand{\Pinner}{\subinner{\Pproj}}
\providecommand{\Pouter}{\subouter{\Pproj}}
\DeclareMathOperator{\argmax}{arg\,max}
\providecommand{\Aineq}{B}
\providecommand{\Aeq}{A}
\providecommand{\bineq}{u}
\providecommand{\beq}{t}
\DeclareMathOperator{\area}{area}
\newcommand{\contact}[1]{\Cspace_{#1}}
\newcommand{\feasible}[1]{\Fspace_{#1}}
\newcommand{\dd}{\; \mathrm{d}}
\newcommand{\figwid}{0.22\columnwidth}

\DeclareMathOperator{\atan2}{atan2}


\newtheorem{theorem}{Theorem}
\newtheorem{definition}[theorem]{Definition}
\newtheorem{lemma}[theorem]{Lemma}
\begin{document}


%%%%%%%%%%%%%% For debugging purposes, I like to display the TOC
%    \tableofcontents
%    \setcounter{tocdepth}{4}
%    \newpage
%%%%%% END TOC %%%%%%%%%%%%%%%%%%%%%%%%%%%%%%%%%%%%%%%

\title{\LARGE \bf 
Design and Implementation of a 3-Axis Biopsy Robot Powered, Imaged, and Controlled by MRI
}
\author{Aaron Becker, Ouajdi Felfoul, Remy Kaldwy, and Pierre E.\ Dupont%, 
\thanks{{A.~Becker, O.~Felfoul, and P.~E.~Dupont are with the Department of Cardiovascular Surgery,  Boston Children's Hospital and Harvard Medical School, Boston, MA, 02115 USA {\tt\small first name.lastname@childrens.harvard.edu}. R.~Kaldawy  is a high school  senior at Boston University Academy, Boston, MA 02215 USA {\tt\small Remy_Kaldawy@buacademy.org}. This work was supported by the National Science Foundation under
\href{http://nsf.gov/awardsearch/showAward?AWD_ID=1208509}{IIS-1208509} and by the \href{http://wyss.harvard.edu/}{Wyss Institute for Biologically Inspired Engineering}.  
}
} %\end thanks
} % end author block

%\author{Michael~Shell,~\IEEEmembership{Member,~IEEE,}
%        John~Doe,~\IEEEmembership{Fellow,~OSA,}
%        and~Jane~Doe,~\IEEEmembership{Life~Fellow,~IEEE}% <-this % stops a space
%\thanks{M. Shell is with the Department
%of Electrical and Computer Engineering, Georgia Institute of Technology, Atlanta,
%GA, 30332 USA e-mail: (see http://www.michaelshell.org/contact.html).}% <-this % stops a space
%\thanks{J. Doe and J. Doe are with Anonymous University.}% <-this % stops a space
%\thanks{Manuscript received April 19, 2005; revised December 27, 2012.}}


\maketitle

\begin{abstract}
Actuators that are powered, imaged, and controlled by Magnetic Resonance (MR) scanners can inexpensively provide wireless control of MR-guided robots. Similar to traditional electric motors, the MR scanner acts as the stator and generates propulsive torques on actuator rotors containing one or more ferrous particles. This paper designs and implements a needle biopsy robot operated by three MRI actuators
This paper provides detail on designing, imaging, and controlling a multiple-DOF MRI actuator. 
\end{abstract}

%##################################################################    
\section{Introduction}\label{sec:Intro}
%##################################################################    
 \IEEEPARstart{T}{his} paper presents progress on a three-DOF needle biopsy robot wirelessly powered, imaged, and controlled by an unaltered clinical MR scanner. 
 Robots can be powered, imaged, and controlled using Magnetic Resonance Imaging (MRI) scanners  \cite{Vartholomeos2012,Vartholomeos2013,Chanu2008,Felfoul2014}.  This paper implements the control techniques for simultaneous control of MRI actuators proposed in \cite{Becker2014}.

Any ferrous material placed inside the bore of an MR scanner becomes strongly magnetized.  The three orthogonal gradient magnetic fields inside the MRI can exert forces on the ferrous material.  Though these gradient fields are controllable, they are relatively weak: the maximum force produced by a 40mT gradient field on a fully magnetized steel particle is only 71\% the force of gravity. These maximum gradients require large current that cause heating of the coils and therefore cannot be applied for long durations. On a Skyra 3T MR scanner, the maximum gradient that can be constantly applied is 23 mT, which is 41\% of the force of gravity on the same particle. 

The force produced by the gradient fields can be magnified through a gear train.

    \begin{figure}
 \centering
\begin{overpic}[width = \columnwidth]{Biopsyrobot.JPG}
\end{overpic}
\caption{\label{fig:PrettyPhotoOfBiopsyRobot}The 3-DOF needle-biopsy robot described in this paper.}
\end{figure}


    
%##################################################################    
\section{Related Work}\label{sec:RelatedWork}
%##################################################################    
   
\subsection{MRI-powered robotics}
    moving capsules \cite{Martel2007}\\
    
 controlling multiple capsules with MRI\cite{Vartholomeos2012}
   
\subsection{MRI-compatible Robots}
   
\subsection{Needle Biopsy Robots}
Needle insertion requires \textcolor{red}{xx} N of force (porcupine study \cite{Cho26122012}, Dupont needle insertion \cite{Mahvash2010})
   
Previous work presented an MRI actuator that could drive a needle with up to 9.7 N  of force\cite{Felfoul2014}.
   

   


%##################################################################       
\section{Theory}\label{sec:Theory}
%##################################################################    

The theory of MRI multi-actuator control was introduced in \cite{Becker2014}, and based on work on work with single-axis MRI actuators \cite{Bergeles2013,Vartholomeos2011}. This section quickly reviews the theory, and then designs controllers for the biopsy robot with three orthogonal rotors.

A ferrous particle in the strong static field of an MRI becomes magnetized, and its magnetization magnitude asymptotically approaches the saturation magnetization $\mathbf{M}_s$ per unit volume of the material.  The MRI gradient coils  produce a magnetic field  $\mathbf{B}_g(t)$. This field exerts 
 on the ferrous particle the force 
\begin{equation}
\mathbf{F}(t) = v\left( \mathbf{M}_s
\cdot \nabla \right) \mathbf{B}_g(t). \label{eq:forceOnDipole}
\end{equation}
Here $v$ is the magnetic volume of the material.  The magnetic field $\mathbf{B}_g(t)$ is designed to produce three independent gradients:
\begin{equation}
\left[ F_x,F_y, F_z \right]^\intercal\!\!(t)= v M_{sz}\left[ 
   \frac{ \partial B_{gz}}{\partial x}, 
   \frac{ \partial B_{gz}}{\partial y},
   \frac{ \partial B_{gz}}{\partial z} 
   \right]^\intercal\!\!\!\!(t)
\label{eq:applicableForces}
\end{equation}
Here it has been reasonably assumed that $M_{sz} \gg M_{sx}, M_{sy}$.
These gradients apply three independent forces on any ferromagnetic spheres inside the MRI.  
 \begin{figure}
 \centering
% \vspace{-1em}
\begin{overpic}[width = 0.85\columnwidth]{rotor.JPG}\end{overpic}
 \vspace{-1em}
\caption{
\label{fig:Schematic}
MRI-powered, single-DOF rotor with gear for power transmission.
}
\vspace{-1.5em}
\end{figure}
This paper investigates rotors that constrain the $i$th ferromagnetic sphere to rotate about an axis $\mathbf{a}_i$ with a moment arm of length  $r_i$, as shown in Fig.~\ref{fig:Schematic}.  The rotor's configuration is fully described  by its angular position and velocity $[\theta_i, \dot{\theta}_i]^\intercal$. The configuration space of all $n$ rotors is $\R^{2\times n}$,  and the dynamic equations are

\begin{equation}
J_i\ddot{\theta}_i(t) = -b_i\dot{\theta}_i(t) -\tau_{f_i}-\tau_{\ell_i} + r_i \mathbf{F}\cdot \mathbf{p}_i(t)
\label{eq:rotorDynamics}
\end{equation}
Here $J_i$ is the moment of inertia, $b_i$ the coefficient of viscous friction,  $\tau_{f_i}$ the summation of all non-viscous friction terms seen by the input,  and $\tau_{\ell_i}$ the load torque. The rotor torque is the magnetic force projected  to a vector tangent to the ferrous sphere's positive direction of motion, $r_i \mathbf{F}\cdot \mathbf{p}_i(t)$
Actuator torque is maximized when $\mathbf{F}(t) = g_{M}V M_{sz}\sgn(\mathbf{p}_i(t)) $, where  $g_{M}$ is the maximum gradient.

\subsection{Open-loop control}\label{subsec:TheoryOpenLoop}

\subsection{Closed-loop control}\label{subsec:TheoryClosedLoop}

\subsection{Interleaving Sensing and Actuation}\label{subsec:InterleaveSenseActuate}



  
 
%##################################################################    
\section{Design}
%##################################################################    
 
 The robot prototype is mostly 3D-printed, making the robot easily replicated at other institutions and hospitals with an MRI scanner. The 3D printed components were designed in \href{http://www.solidworks.com/}{SolidWorks}, and are available at \href{http://www.thingiverse.com/thing:449517}{thingiverse.com}\footnote{\href{http://www.thingiverse.com/thing:449517}{www.thingiverse.com/thing:449517}}. This illustrates how robot designs can be easily shared and working robots can be printed by third parties.  This is similar to recent efforts to make printable robots, e.g. \cite{Felton08082014}.  These efforts have created self-folding structures, but the motors and controllers are commercial devices that are currently manually added.  In contrast, in this design the MRI scanner performs the difficult tasks of actuation and sensing, while the robot consists of a plastic 3D-printed structure and only requires adding a ferrous sphere for each actuator.

\subsection{Motor Design}
\todoRemy{Contrasting architectures: linear vs rotary actuators.  
Needle driving: tendons vs timing belts vs friction rollers vs gear system.  
Needle replacement.  
Motor modularity.
Required forces. }


To study the preceding topics in the context of a practical example system, a three-axis biopsy robot powered by DC motors was considered~\cite{Walsh2010}.  Figure \ref{fig:PrettyPhotoOfBiopsyRobot} shows a prototype that is powered and actuated by an MRI scanner.  The system has a fixed base that is attached to the patient.   Two actuators, $\theta_x$ and $\theta_y$, control orthogonal axes of a nested spherical yoke.  A carriage rides along the intersection of this yoke.  Mounted on this carriage is a third actuator, $\theta_{needle}$,  that can insert a needle through a pivot point at the center of the base.  The original design allows $\theta_x$ and $\theta_y$ to rotate between $[-\pi/6,\pi/6]$, and inserts $\theta_{needle}$ from [0,100]mm, resulting in a  spherical quadrilateral workspace with volume $(\theta_{needle})^3\pi/9$.  The base has diameter 100m, and the nested yokes have radii 50mm.

The rotors attached to the spherical yokes are subject to a torque due to gravity, $\tau_{mass}$, and a needle-depth dependent torque, $\tau_{needle}$, as well as the frictional torques in \eqref{eq:rotorDynamics}.
\begin{align}
\tau_{mass} &= m  g  \ell  \sin(\theta_i) &\mathrm{[Nmm]} \nonumber \\
\tau_{needle} &= \theta_{needle} \frac{60}{100} &\mathrm{[Nmm]} 
\end{align}
The lumped mass $g$ of the carriage and needle actuator is 0.1kg and the yoke radius $\ell$=50mm~\cite{Walsh2010}.  The frictional losses in the gear train are represented by $\eta_e \in[0,1]$, a dimensionless parameter for motor efficiency.  

The two yoke actuators require 100Nmm of torque and the needle actuator requires 50Nmm of torque.  Assuming a conservative $\eta_e=0.5$, with a gear reduction ratio $G$, the motor torque is
\begin{align}
\tau_{motor}  =  \eta_\tau  \eta_e G r_i \frac{4}{3} \pi r_{sphere}^3 M_{sz} g_{M}  . \label{eq:RotorForceGear}
\end{align}
Here $\eta_\tau$ is the average torque per rotor for $n=3$ rotors, which was evaluated as $\approx$0.81 in   \cite{Becker2014}. 
Using a rotor radius of $r_i$=20mm and a sphere radius $r_{sphere}$=6mm, the yoke actuators require a gear reduction of $G_{xy}$=250, and the needle actuator requires $G_{n}$=125. 


The ferrous spheres must be separated to minimize dipole-dipole forces, as will be described in Section \ref{subsec:RotorSpacing}. According to \eqref{eq:dipoledipolePercentGrad}, to limit dipole-dipole forces to less than 10\% of maximum gradient forces with 6mm radius ferrous spheres requires at least 116mm spacing. 


\subsection{Optimizing Spacing}\label{subsec:RotorSpacing}

There are several concerns when placing MRI actuators near each other: magnetic attraction between ferrous particles,  linearly separable MRI fiducials, and minimizing the influence of  ball-bearings on other actuator's fiducials. 

\paragraph{Fiducial offset}
The MRI fiducials must be accurately placed in the magnetic field generated by the ferrous ball bearing. The magnetic field is given by \eqref{eq:dipoleMagField}, with $\mathbf{m}_1 = [0,0,4/3 \pi r_s^3 M_{sat}]^\top$.  The fiducials should be placed where the magnetic field equals the ratio of the RF frequency offset $f_{offset}$, and the gyromagnetic ratio $\gamma = 42.48$MHz.
\begin{align}\label{eq:FiducialFerrousSpacing}
 \mathbf{B}_{\mathbf{r}_1}(x,z) =& \frac{f}{\gamma}\nonumber \\
d_{xy} =& \sqrt[3]{\frac{M_{sat} r_s^3 \mu_ 0 \gamma }{3 f_{offset}}} \nonumber \\
d_{z} =& \sqrt[3]{2\frac{M_{sat} r_s^3 \mu_ 0 \gamma }{3 f_{offset}}}
\end{align}
For $r = 2.5$mm alloy steel spheres, $d_{xy} \approx 0.047$mm and $d_{z} \approx 0.060$mm. $d_{z} = \sqrt[3]{2} d_{xy} $, and scale linearly with the ferrous sphere radius. The $z$ axis actuator can only be imaged using positive frequency RF signals, and the $x$ and $y$ actuators must be imaged using negative RF signals, as shown in Fig.~\ref{fig:ArrangingMRspots}.


 \begin{figure}
 \centering
 \begin{overpic}[width = 0.47\columnwidth]{PosAndNegFiducial}\end{overpic}\begin{overpic}[width = 0.47\columnwidth]{ArrangingMRspots}\end{overpic}
\caption{\label{fig:ArrangingMRspots}To be imaged, MR fiducials (yellow) must be accurately placed in the magnetic field generated by the ferrous ball-bearing.  Positive RF frequencies illuminate fiducial markers along the $z$ axis (red) and negative RF frequencies illuminate fiducials in the $xy$ plane (blue).  These fiducial markers are optimally placed perpendicular to the rotor axis ($\perp$), not parallel ($\parallel$), because this places more fiducial liquid at the center RF frequency (dashed line).}
\end{figure}


\paragraph{Ferrous-Ferrous spacing}
 Each rotor contains a ferrous ball-bearing, which will attract and repulse each other.  These bearings must be placed far enough apart such that the forces exerted are negligible compared to the magnetic gradient.
 
 Any ferrous material placed in the magnetic field of an MR scanner becomes a strong magnetic dipole.  With multiple MR-powered motors, these dipoles exert forces on each other.  Dipole forces overpower MRI gradient forces if rotors are closer than a threshold distance.

The magnetic field at position $\mathbf{r}_2$ generated by a spherical magnet at position $\mathbf{r}_1$ with magnetization  $\mathbf{m}_1$ is  \cite{Schill2003} %\cite{thomaszewski2008magnets}
\begin{align}
\label{eq:dipoleMagField}
 \mathbf{B}_{\mathbf{r}_1}(\mathbf{r_2}) = \frac{\mu_0}{4 \pi}\frac{3 \mathbf{n}_{12}(\mathbf{n}_{12} \cdot \mathbf{m}_1) - \mathbf{m}_1}
 {|\mathbf{r}_2-\mathbf{r}_1|^3},
\end{align}
with  $\mathbf{n}_{12} = (\mathbf{r}_2-\mathbf{r}_1)/|\mathbf{r}_2-\mathbf{r}_1|$. This is the \emph{magnetic field of a dipole}.
 The force applied to a dipole at $\mathbf{r}_1$ with magnetic moment $\mathbf{m}_1$ by another dipole at $\mathbf{r}_2$ with magnetic moment $\mathbf{m}_2$ is approximated by
\begin{align}
\mathbf{F}_{12} \approx \frac{3\mu_0}{4 \pi} \frac{1}{|\mathbf{r}_2 - \mathbf{r}_1 |^4}
\left[5 \mathbf{n}_{12}\Big(\left(\mathbf{m}_1 \cdot \mathbf{n}_{12} \right)   \left(\mathbf{m}_2 \cdot \mathbf{n}_{12} \right) \Big) \right. \nonumber \\
\left.
-  \mathbf{n}_{12} \left(\mathbf{m}_2 \cdot \mathbf{m}_1 \right)
-  \mathbf{m}_{1} \left(\mathbf{m}_2 \cdot \mathbf{n}_{12} \right)  -  \mathbf{m}_{2} \left(\mathbf{m}_1 \cdot \mathbf{n}_{12}\right)   \right].\nonumber
\label{eq:dipoleForce}
\end{align}

 \begin{figure}
 \centering
\begin{overpic}[height = 0.4\columnwidth]{MagneticDipoleField2p5mmRast.pdf}
\tiny
\put(28,29){0}
\put(28,73){0}
\put(89,73){0}
\put(89,29){0}
\put(27,51){-0.1$g_{M}$}
\put(52,17){0.1$g_{M}$}
\put(56,38){$g_{M}$}
\put(64.5,52){-$g_{M}$}
\small
\put(35,-8){$r_{sphere}$ = 2.5mm}
\end{overpic}\begin{overpic}[height = 0.4\columnwidth]{SeparationForGradientForcesr2p5mm.pdf}\end{overpic}
\caption{\label{fig:MagneticDipoleField3mmRast}One ferrous sphere in a 3T magnetic field exerts a  force $\mathbf{F}$ on an identical sphere.   The contour lines show $\mathbf{F}\cdot \mathbf{n}_{12}$, the force component radially outward from the sphere at $(0,0)$ compared to the maximum force provided by the gradient coils $g_{M}$.  This force is attractive (red) along the $z$-axis and repulsive (blue) perpendicular to $z$. The magnetic field is symmetric about the $z$-axis.  If two spheres move within the dark red region, they cannot be separated using the gradient field. }
\end{figure}
Figure \ref{fig:MagneticDipoleField3mmRast} shows contour plots for the magnetic force exerted by two identical spheres on each other.  The contour lines are drawn at $\mathbf{F}_{12}\cdot \mathbf{n}_{12} = g_{M}\cdot \{-1,-\frac{1}{10},0,\frac{1}{10},1\}$.  The maximum force is along the $z$-axis

\begin{equation}
F_{\text{attraction}} = -\frac{ 8 M_{s}^2 \mu_0 \pi  r^3_1 r^3_2 }{3 d^4},
\label{eq:attractionForce}
\end{equation}
where $d$ is the distance separating two sphere of radius $r_1$ and $r_2$, each with magnetic saturation $M_{s}$. For steel, $M_s$=1.36$\times10^6$. 
The vacuum permeability $\mu_0$ is, by definition, $4 \pi\times 10^{-7}$ V$\cdot$ s/(A$\cdot$m).

The critical distance when the attractive force becomes greater than the maximum gradient force is $\sqrt[4]{\frac{2  \mathbf{M}_s \mu_0 r^3_{sphere} }{g_{M}}}. $  This interaction decays quickly and at distance $\approx 5.4 r_{sphere}^{3/4}$ is  10\% of the maximum gradient. The required distance, $d$, to ensure dipole-dipole forces are less than some $percentage$ of the maximum gradient is given by
%but rather the distance at which you can ignore the interaction - e.g., it only reduces max torque by 10\%. 
\begin{equation}
d \ge \sqrt[4]{\frac{ 2 \frac{100}{percentage} M_{sz} \mu_0 r^3_{sphere} }{g_{M}}}.
\label{eq:dipoledipolePercentGrad}
\end{equation}





\paragraph{Fiducial-Ferrous spacing}
Moreover, fiducials must be separated from other ferrous ball-bearings.  The magnetic field from ball bearings is additive, so ferrous ball-bearings  must be spaced to avoid influencing non-associated MRI fiducials.

As shown in Fig.~\ref{fig:MagneticFieldCapsuleRange}, if other ferrous particles are too close a fiducial may no longer be illuminated.  
\eqref{eq:dipoleMagField}, 
The ideal spacing between fiducial and ferrous particle is given by \ref{eq:FiducialFerrousSpacing}, and is defined as the distance where the magnetic field value equals the RF frequency offset.  The MRI fiducials used have a radius of $r_{\rm{fiducial}}=$3.5mm.  If the influence of another ferrous particle moves the magnetic field more than $r_{\rm{fid}}$, the fiducial will be insufficiently illuminated.
 The minimum separation is found by solving for $d_{xy\rm{Min}}$ and $d_{z\rm{Min}}$:
\begin{align}\label{eq:FiducialFerrousFerrousSpacing}
 \mathbf{B}_{(0,0)}(d_{xy}+r_{\rm{fid}},0) +  \mathbf{B}_{(d_{xy\rm{Min}},0)}(d_{xy}+r_{\rm{fid}},0) =& \frac{f_{offset}}{\gamma}\nonumber \\
  \mathbf{B}_{(0,0)}(0,d_z+r_{\rm{fid}}) +  \mathbf{B}_{(0,d_{z\rm{Min}})}(0,d_z+r_{\rm{fid}}) =& \frac{f_{offset}}{\gamma} 
\end{align}
For $r_s=2.5$mm, $d_{xy\rm{Min}}\approx13$mm and $d_{z\rm{Min}}17\approx$mm.  A diamond shaped polygonal region with vertices at  $(\pm d_{xy\rm{Min}},0),(0,\pm d_{z\rm{Min}})$ conservatively bounds the unsafe region, as shown in Fig.~\ref{fig:MagneticFieldCapsuleRange}.
 \begin{figure*}
 \centering
\newcommand{\figheight}{0.28\columnwidth}
\scriptsize
\begin{overpic}[width = \figheight]{MagneticFieldCapsuleRange1}\put(25,-5){$x>d_{xy\rm{Min}}$}\end{overpic}
\begin{overpic}[width =\figheight]{MagneticFieldCapsuleRange2}\put(25,-5){$x=d_{xy\rm{Min}}$}\end{overpic}
\begin{overpic}[width =\figheight]{MagneticFieldCapsuleRange3}\put(25,-5){$x<d_{xy\rm{Min}}$}\end{overpic}
\begin{overpic}[width =\figheight]{MagneticFieldCapsuleRange4}\put(25,-5){$(x,y)$ safe}\end{overpic}
\begin{overpic}[width =\figheight]{MagneticFieldCapsuleRange5}\put(25,-5){$z>d_{z\rm{Min}}$}\end{overpic}
\begin{overpic}[width =\figheight]{MagneticFieldCapsuleRange6}\put(25,-5){$z=d_{z\rm{Min}}$}\end{overpic}
\begin{overpic}[width =\figheight]{MagneticFieldCapsuleRange7}\put(25,-5){$z<d_{z\rm{Min}}$}\end{overpic}
\caption{\label{fig:MagneticFieldCapsuleRange}The magnetic field due to ferrous particles is additive.  The MRI fiducials are placed either $d_x$ or $d_{xy}$ distance from their corresponding ferrous particle to be illuminated by a certain RF frequency (dashed line).  If other ferrous particles are too close, the fiducial will no longer be illuminated.  The orange polygon conservatively defines this unsafe region. }
\end{figure*}




\paragraph{Fiducial-Fiducial spacing}
The actuator axis and fiducials must be spaced such that their projections onto the $x$, $y$, and $z$ axis are linearly separable. The fiducial projections are given in Table \ref{tab:actuatorprojections} using the trigonometric shorthand $\cos(\theta) = c_\theta$, $\sin(\theta) = s_\theta$.
The z-axis rotors are imaged separately using a positive frequency, leaving three constraints:
\begin{align}
x_x &\not\in x_y +d_{xy} \pm r \nonumber \\
y_x &\not\in y_y -d_{xy} \pm r \nonumber \\
z_x &\not\in z_y \pm 2r  \label{eq:markerProjectionSpacing}
\end{align}

\begin{table}[h]
\definecolor{light-gray}{gray}{0.95}
\begin{tabular}{ll | lll}
         &                & \multicolumn{3}{c}{projection} \\
axis & sphere center		     & $x$        & $y$        & $z$     \\ \hline  
$x$        &      $[x_x,y_x,z_x]$          &     $x_x - d_{xy}$     &    $y_x+r c{\theta_x}$      &    $z_x+r s_{\theta_x}$          \\
$y$        &      $[x_y,y_y,z_y]$           &     $x_y+r c_{\theta_y}$        &     $y_y - d_{xy}$     &     $z_y+r s_{\theta_y}$         \\
\rowcolor{light-gray} $z$        &       $[x_z,y_z,z_z]$          &     $x_z+r c_{\theta_z}$        &     $y_z+r s_{\theta_z}$         &    $z_z - d_{z}$     
\end{tabular}
\caption{Projections of fiducial position onto $x$,$y$, and $z$ axes.}
\label{tab:actuatorprojections}
\end{table}



%\subsection{MRI}

\subsection{Material Selection}

With the notable exception of the ferrous sphere made of E52100 Alloy Steel, the biopsy robot must be constructed of non-ferrous material.  To avoid both burn dangers and damping forces due to eddy currents, materials must be non-conductive.

The robot is constructed of  Vero White %http://www.stratasys.com/materials/polyjet/rigid-opaque (VeroWhitePlus RGD835) http://www.stratasys.com/~/media/Main/Secure/Material%20Specs%20MS/PolyJet-Material-Specs/PolyJet_Materials_Data_Sheet.pdf
 on an Alaris30 Objet printer %http://www.stratasys.com/3d-printers/design-series/precision/objet30-pro
braided  suture cord for the tendon system, 
 electrical grade fiberglass (GP03, www.mcmaster.com/2591K11) for long structural members

    

%##################################################################    
\section{Experiments}\label{sec:Experiments}
%##################################################################        
    A series of experiments were conducted to quantify the power, efficiency, and speed of the actuator.  The actuators were tested both individually and as a group.
    
    
\subsection{Open-Loop}
\paragraph{Single-actuator actuation}
This experiment used an open-loop sinusoid rotating at XX Hz for XX seconds.  The rotor completed XX spins, turning the 

\paragraph{Independent actuation}

\todoAaron{report motor efficiency }

\paragraph{Maximum speed actuation}


\subsection{Closed-Loop}
\paragraph{Single-actuator actuation}

\paragraph{Independent actuation}
\todoAaron{report motor efficiency }

\paragraph{Maximum speed actuation}

\subsection{Accuracy Test}

\subsection{Needle Imaging}

A 22-gauge Chiba Biopsy Needle\footnote{DCHN-22-20.0-U} was used and under MR control was inserted into a 50mm diameter lime.  A 22-gauge needle has a diameter of 0.72 mm.  A T1 weighted sequence T1-turbo spin echo with TR 1200 ms and TE 9.5ms transversal view. Slice thickness of 2 mm.
INSERT IMAGE OF LIME 

%##################################################################    
\section{Conclusions}\label{sec:Conclusions}
%##################################################################        

A prototype


   
 % \IEEEtriggeratref{3}   %used to manually flush the columns
\bibliographystyle{IEEEtran}
\bibliography{IEEEabrv,../bib/aaronrefs}%,../aaronrefs}
\end{document}

