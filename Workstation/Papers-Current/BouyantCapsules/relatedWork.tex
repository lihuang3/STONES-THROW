

%%%%%%%%%%%%%%%%%%%%%%%%%%%%%%%%%%%%%%%%%%%%%%%%%%%%%%%%%%%
\section{Related Work}\label{sec:RelatedWork}
%%%%%%%%%%%%%%%%%%%%%%%%%%%%%%%%%%%%%%%%%%%%%%%%%%%%%%%%%%%


\subsection{MRI actuators}

\paragraph{Motors}

\paragraph{Milli-robots}

%BOOK TO READ: John A. Pelesko. Self assembly - The Science of Things That Put Themselves Together. Chapman & Hall/CRC, 2007.

%active modules
Zykov et al.~\cite{Zykov2007} built actuated centimeter-cale modules equipped with electromagnets that could selectively control the morphology of the robotic assembly.  


%bridge gap between active and passive
Nap et. al  \cite{Klavins-RSS-06} built populations of magnetic modules with tunable stochastic properties.  By modifying the breaking probabilities they could change the expected proportion of robots in the set of possible configurations.


%passive modules
Vartholomeos \cite{Vartholomeos2012} introduced MRI control of multiple magnetic capsules by varying their inertial properties.  

Diller et al.~\cite{Diller01122011} designed magnetic modules less than 1mm in every dimension that could be steered in 2D with an external magnetic field.  Individual modules could be held in place with a specialized electro-static substrate. 
 In recent work Diller et al.~\cite{diller2013modular} designed neutrally buoyant particles constructed of soft-magnetic material that could be demagnetized, allowing new modules to be docked to the assembly, and then welded with hot-melt adhesive.  The assembly could then be re-magnetized and steered to a desired position.
 
 
 
Alink et al.~\cite{eemcs21309} studied the interactions of spherical magnetic dipoles and modeled the energy landscape for arrangements of 2,3, and 4 modules.  



\subsection{Controlling many robots with uniform inputs}