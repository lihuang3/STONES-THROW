\section{Capsule Design and Control}\label{sec:designAndControl}



 \begin{figure}
\begin{overpic}[width = \columnwidth]{LotOneBouyancySim.pdf}\end{overpic}
\caption{
\label{fig:LotOneBouyancySim}
The first lot of capsules were 5$\times$ scale.  Each was a 10mm capsule containing a 2mm-diameter ferrous ball-bearing.  Varying the shell thickness varied the buoyancy.  Capsules in the white region could be selectively changed from floating to sinking by varying the $x$-gradient (vertical).
}
\end{figure}



 \begin{figure}
\begin{overpic}[width = \columnwidth]{LotOneBouyancySim.pdf}\end{overpic}
\caption{
\label{fig:shellThickness}
Capsules used in these experiments.  A 4 mm steel bead is enclosed by a water-tight capsule of ABS plastic.  Capsules are shown in the unassembled and assembled state.
}
\end{figure}





 \begin{figure}
\begin{overpic}[width = \columnwidth]{LotOneBouyancySim.pdf}\end{overpic}
\caption{
\label{fig:ContainerAndMRI}
(Left)  container vessel.  A molded plastic grid lines the container bottom.  (Right) experimental setup with the vessel and capsules inside a 600mm bore Siemens 3T MR-scanner.
}
\end{figure}