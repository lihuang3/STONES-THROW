\section{Introduction}\label{sec:Intro}
%
 %function and implementation function
 
This project addresses the need for ultra-minimally invasive robots to perform diagnostic and therapeutic tasks deep inside the body while simultaneously addressing the need for integrated real-time imaging. In contrast, most existing approaches employ large robots and consequently require relatively large incisions.  In consequence, substantial healthy tissue can be damages en route to the diseased tissue.  Because of this collateral damage, treatment is often delayed or alternative less-effective techniques are employed.  For example, in brain surgery, it is often necessary to sacrifice functioning brain tissue to reach an underlying tumor.  In  heart disease, less-effective catheterization procedures may be performed to avoid the risks and trauma of open-heart surgery. \todo{ref?}
 
 At the millimeter and sub millimeter scale, groups of MRI-powered swimming robots could perform targeted therapies inside fluid-filled regions of the body, such as the ventricular system of the brain. \todo{figure from NRI?} Because the ventricles provide access to a substantial portion of the brain, the proposed millirobots, injected into the spine and steered to the brain, could significantly reduce the morbidity of current procedures whiles also enabling a broad range of new procedures.  These millirobots could be capable of performing localized therapies such as drug and cell delivery for the treatment of cancer, epilepsy and other diseases.  They could also be capable of forming sensor networks to monitor such quantities as pressure.  As delivery vehicles(of drugs, for example), they would be superior to systemic delivery since they would enable high concentrations to be delivered to very specific locations while bypassing the blood-brain barrier and without exposing the rest of the body.
 
 
 
 
 Math about swimmer bouyancy
 
 Math about dipoles in a very strong magnetic field \cite{thomaszewski2008magnets}
 
 Simulation of many dipoles in a strong magnetic field
 -- 4 dipoles in field, start at random positions, see what shapes are formed.  Do experiment 1000s of times and get shape formation probabilities.
We derive inspiration from simulations by Alink et al.~on self-assembly of 3 to 4 permanent magnets\cite{alink2011simulating}, and simulations of aggregation with nanoparticles in \cite{vartholomeos2010simulation}.
 
 Experiments -- place 2-5 magnets in water,  record ending configurations.  Show this matches simulation (?)
 
 Controlled buoyancy experiments -- show we can make configurations that are unlikely by chance
 
 Show we can make a desired configuration
 
 Steer the assembly around.
 
 
 \subsection{Why MRI?}

 

 Our paper is organized as follows.  After a discussion of related work in Section \ref{sec:RelatedWork}, we describe our model and control law in Section \ref{sec:designAndControl}.  Section \ref{sec:analysis} examines how to optimize system design.   We report the results of our experiments in Section \ref{sec:experiment}, and end with concluding remarks in Section \ref{sec:conclusion}.


