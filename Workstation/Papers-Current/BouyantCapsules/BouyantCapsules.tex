% to submithttps://ras.papercept.net/conferences/scripts-n/start.pl

\documentclass[letterpaper, 10 pt, conference]{ieeeconf}
\IEEEoverridecommandlockouts
\usepackage{calc}
\usepackage{url}
\usepackage{hyperref}
\hypersetup{
  colorlinks =true,
  urlcolor = black,
  linkcolor = black
}
\usepackage{graphicx}
\usepackage[cmex10]{amsmath}
\usepackage{amssymb}
\usepackage{rotating}

\usepackage{nicefrac}
\usepackage{cite}
\usepackage[caption=false,font=footnotesize]{subfig}
\usepackage[usenames, dvipsnames]{color}
\usepackage{colortbl}
\usepackage{overpic}
\graphicspath{{./pictures/pdf/},{./pictures/ps/},{./pictures/png/},{./pictures/jpg/}}
\usepackage{breqn} %for breaking equations automatically
\usepackage[ruled]{algorithm}
\usepackage{algpseudocode}
\usepackage{multirow}


\newcommand{\todo}[1]{\vspace{5 mm}\par \noindent \framebox{\begin{minipage}[c]{0.98 \columnwidth} \ttfamily\flushleft \textcolor{red}{#1}\end{minipage}}\vspace{5 mm}\par}
% uncomment this to hide all red todos
%\renewcommand{\todo}{}

%% ABBREVIATIONS
\newcommand{\qstart}{q_{\text{start}}}
\newcommand{\qgoal}{q_{\text{goal}}}
\newcommand{\pstart}{p_{\text{start}}}
\newcommand{\pgoal}{p_{\text{goal}}}
\newcommand{\xstart}{x_{\text{start}}}
\newcommand{\xgoal}{x_{\text{goal}}}
\newcommand{\ystart}{y_{\text{start}}}
\newcommand{\ygoal}{y_{\text{goal}}}
\newcommand{\gammastart}{\gamma_{\text{start}}}
\newcommand{\gammagoal}{\gamma_{\text{goal}}}
\providecommand{\proc}[1]{\textsc{#1}}


\newcommand{\ARLfull}{Aero\-space Ro\-bot\-ics La\-bora\-tory }
\newcommand{\ARL}{\textsc{arl}}
\newcommand{\JPL}{\textsc{jpl}}
\newcommand{\PRM}{\textsc{prm}}
\newcommand{\CM}{\textsc{cm}}
\newcommand{\SVM}{\textsc{svm}}
\newcommand{\NN}{\textsc{nn}}
\newcommand{\prm}{\textsc{prm}}
\newcommand{\lemur}{\textsc{lemur}}
\newcommand{\Lemur}{\textsc{Lemur}}
\newcommand{\LP}{\textsc{lp}} 
\newcommand{\SOCP}{\textsc{socp}}
\newcommand{\SDP}{\textsc{sdp}}
\newcommand{\NP}{\textsc{np}}
\newcommand{\SAT}{\textsc{sat}}
\newcommand{\LMI}{\textsc{lmi}}
\newcommand{\hrp}{\textsc{hrp\nobreakdash-2}}
\newcommand{\DOF}{\textsc{dof}}
\newcommand{\UIUC}{\textsc{uiuc}}
%% MACROS


\providecommand{\abs}[1]{\left\lvert#1\right\rvert}
\providecommand{\norm}[1]{\left\lVert#1\right\rVert}
\providecommand{\normn}[2]{\left\lVert#1\right\rVert_#2}
\providecommand{\dualnorm}[1]{\norm{#1}_\ast}
\providecommand{\dualnormn}[2]{\norm{#1}_{#2\ast}}
\providecommand{\set}[1]{\lbrace\,#1\,\rbrace}
\providecommand{\cset}[2]{\lbrace\,{#1}\nobreak\mid\nobreak{#2}\,\rbrace}
\providecommand{\lscal}{<}
\providecommand{\gscal}{>}
\providecommand{\lvect}{\prec}
\providecommand{\gvect}{\succ}
\providecommand{\leqscal}{\leq}
\providecommand{\geqscal}{\geq}
\providecommand{\leqvect}{\preceq}
\providecommand{\geqvect}{\succeq}
\providecommand{\onevect}{\mathbf{1}}
\providecommand{\zerovect}{\mathbf{0}}
\providecommand{\field}[1]{\mathbb{#1}}
\providecommand{\C}{\field{C}}
\providecommand{\R}{\field{R}}
\newcommand{\Cspace}{\mathcal{Q}}
\newcommand{\Uspace}{\mathcal{U}}
\providecommand{\Fspace}{\Cspace_\text{free}}
\providecommand{\Hcal}{$\mathcal{H}$}
\providecommand{\Vcal}{$\mathcal{V}$}
\DeclareMathOperator{\conv}{conv}
\DeclareMathOperator{\cone}{cone}
\DeclareMathOperator{\homog}{homog}
\DeclareMathOperator{\domain}{dom}
\DeclareMathOperator{\range}{range}
\DeclareMathOperator{\sign}{sgn}
\providecommand{\polar}{\triangle}
\providecommand{\ainner}{\underline{a}}
\providecommand{\aouter}{\overline{a}}
\providecommand{\binner}{\underline{b}}
\providecommand{\bouter}{\overline{b}}
\newcommand{\D}{\nobreakdash-\textsc{d}}
%\newcommand{\Fspace}{\mathcal{F}}
\providecommand{\Fspace}{\Cspace_\text{free}}
\providecommand{\free}{\text{\{}\mathsf{free}\text{\}}}
\providecommand{\iff}{\Leftrightarrow}
\providecommand{\subinner}[1]{#1_{\text{inner}}}
\providecommand{\subouter}[1]{#1_{\text{outer}}}
\providecommand{\Ppoly}{\mathcal{X}}
\providecommand{\Pproj}{\mathcal{Y}}
\providecommand{\Pinner}{\subinner{\Pproj}}
\providecommand{\Pouter}{\subouter{\Pproj}}
\DeclareMathOperator{\argmax}{arg\,max}
\providecommand{\Aineq}{B}
\providecommand{\Aeq}{A}
\providecommand{\bineq}{u}
\providecommand{\beq}{t}
\DeclareMathOperator{\area}{area}
\newcommand{\contact}[1]{\Cspace_{#1}}
\newcommand{\feasible}[1]{\Fspace_{#1}}
\newcommand{\dd}{\; \mathrm{d}}
\newcommand{\figwid}{0.22\columnwidth}

\DeclareMathOperator{\atan2}{atan2}


\newtheorem{theorem}{Theorem}
\newtheorem{definition}[theorem]{Definition}
\newtheorem{lemma}[theorem]{Lemma}
\begin{document}

%%%%%%%%%%%%%% For debugging purposes, I like to display the TOC
%    \tableofcontents
%    \setcounter{tocdepth}{3}
%\newpage
%\mbox{}
%\newpage
%\mbox{}
%\newpage

%%%%%% END TOC %%%%%%%%%%%%%%%%%%%%%%%%%%%%%%%%%%%%%%%

\title{\LARGE \bf 
Directed Self-Assembly of Selectively-Buoyant Capsules\ with a Clinical MRI scanner
%Steering Many Selectively-Buoyant Capsules with a Clinical MRI scanner
%Simultaneous, Multi-axis Control of $n$ Motors with a Clinical MRI Scanner
%how can this title be exciting?
}
\author{Ouajdi Felfoul, Aaron Becker,  Alina Eqtami, and Pierre Dupont%, 
\thanks{{O. Felfoul, A. Becker, A. Eqtami, and P. Dupont are with the Department of Cardiovascular Surgery,  Boston Children's Hospital and Harvard Medical School, Boston, MA, 02115 USA {\tt\small first name.lastname@childrens.harvard.edu}.
}
} %\end thanks
} % end author block
\maketitle

\begin{abstract}
Clinical MRI systems have three magnetic  coil pairs that can create a 3D linearly-varying gradient field.  These gradient fields are relatively weak compared to the static magnetic field, so the magnetic field orientation remains constant.
We engineer a range of selectively buoyant capsules.  By manipulating the vertical magnetic gradient, we can transition these capsules between floating and sunken states.  In this paper we present a control method to steer multiple capsules in optimum time.





\end{abstract}


%%%%%%%%%%% PAPER OUTLINE

%%%%%%%%%%%%%%%
\section{Introduction}\label{sec:Intro}

Robotics offers important contributions to image-guided, minimally invasive surgery.  
Among imaging techniques,  MRI has several advantages.  MRI provides high resolution soft-tissue imaging and does not use ionizing radiation. 
%
%\todo{
%MRI-compatible robotics:  eight DOF  pneumatically actuated robot for aortic valve replacement \cite{li2011pneumatic}
%Minimally-invasive, MRI-compatible trunk robots actuated with shape memory alloy  wires for inter-cranial tumor resection \cite{Ho2012TRO}. %Jaydev Desai
%See \cite{fischer2008mri} for an overview of MRI-compatible actuation techniques.
%}

 MRI  image-guided procedures, however, pose several challenges for robotics\cite{Ho2012TRO,fischer2008mri,Martel2007,Vartholomeos2011,Vartholomeos2013}.  First, all ferrous materials create imaging artifacts.  Ferrous materials must be isolated from the imaging region of interest.  Moreover, the magnetic fields used in an MRI induce forces on any ferrous materials in the robot and turn these materials into strong magnetic dipoles that exert forces on each other.   MRI gradients induce current in any conducting materials, which can generate dangerous amounts of heat and also exert forces. 
 
Despite these challenges, there are a number of recent innovations demonstrating tetherless and inexpensive actuation imaged, powered, and controlled using MRI.   Martel et al. demonstrated in vivo motion control of a mm-scale particle in the carotid artery of swine \cite{Martel2007}. Vartholomeos et al.~designed a single-DOF MRI-powered actuator for use as a tetherless biopsy robot \cite{Vartholomeos2011}.  This was extended to closed-loop control of a single rotor in~\cite{Vartholomeos2013}. Since such results require only scanner software and inexpensive actuator components, dissemination of MRI-based robotic technology has the potential to be rapid and inexpensive.


The contribution of this paper is to develop feedback-control techniques enabling precision control of a rotors. Section \ref{sec:controlLaw} describes an MRI actuator model and control law.    Section \ref{sec:conclusion} ends with concluding remarks.
  
  
 
%   \begin{figure}
%\begin{overpic}[width = \columnwidth]{MRIandRobotBiopsys}
%\put(8,-5){(a) MRI scanner}
%\put(48,-5){(b) 3 independent actuators}
%\end{overpic}
%\caption{
%\label{fig:3orthogonalRotors}
%This paper proves that $n$ non-parallel rotors can be independently actuated by the same magnetic gradient field. The torque from these rotors could power multi-DOF robotic actuators running untethered inside an MRI bore  such as (b), a biopsy robot inspired by~\cite{Walsh2010} that can insert a needle and tilt the needle to a two DOF compound angle.
%}
%\end{figure}


%%%%%%%%%%%%%%%
%%%%%%%%%%%%%%%


%%%%%%%%%%%%%%%%%%%%%%%%%%%%%%%%%%%%%%%%%%%%%%%%%%%%%%%%%%%
\section{Related Work}\label{sec:RelatedWork}
%%%%%%%%%%%%%%%%%%%%%%%%%%%%%%%%%%%%%%%%%%%%%%%%%%%%%%%%%%%


\subsection{MRI actuators}

\paragraph{Motors}

\paragraph{Milli-robots}

%BOOK TO READ: John A. Pelesko. Self assembly - The Science of Things That Put Themselves Together. Chapman & Hall/CRC, 2007.

%active modules
Zykov et al.~\cite{Zykov2007} built actuated centimeter-cale modules equipped with electromagnets that could selectively control the morphology of the robotic assembly.  


%bridge gap between active and passive
Nap et. al  \cite{Klavins-RSS-06} built populations of magnetic modules with tunable stochastic properties.  By modifying the breaking probabilities they could change the expected proportion of robots in the set of possible configurations.


%passive modules
Vartholomeos \cite{Vartholomeos2012} introduced MRI control of multiple magnetic capsules by varying their inertial properties.  

Diller et al.~\cite{Diller01122011} designed magnetic modules less than 1mm in every dimension that could be steered in 2D with an external magnetic field.  Individual modules could be held in place with a specialized electro-static substrate. 
 In recent work Diller et al.~\cite{diller2013modular} designed neutrally buoyant particles constructed of soft-magnetic material that could be demagnetized, allowing new modules to be docked to the assembly, and then welded with hot-melt adhesive.  The assembly could then be re-magnetized and steered to a desired position.
 
 
 
Alink et al.~\cite{eemcs21309} studied the interactions of spherical magnetic dipoles and modeled the energy landscape for arrangements of 2,3, and 4 modules.  



\subsection{Controlling many robots with uniform inputs}
%%%%%%%%%%%%%%%
%%%%%%%%%%%%%%%
\section{Capsule Design and Control}\label{sec:designAndControl}



 \begin{figure}
\begin{overpic}[width = \columnwidth]{LotOneBouyancySim.pdf}\end{overpic}
\caption{
\label{fig:LotOneBouyancySim}
The first lot of capsules were 5$\times$ scale.  Each was a 10mm capsule containing a 2mm-diameter ferrous ball-bearing.  Varying the shell thickness varied the buoyancy.  Capsules in the white region could be selectively changed from floating to sinking by varying the $x$-gradient (vertical).
}
\end{figure}



 \begin{figure}
\begin{overpic}[width = \columnwidth]{LotOneBouyancySim.pdf}\end{overpic}
\caption{
\label{fig:shellThickness}
Capsules used in these experiments.  A 4 mm steel bead is enclosed by a water-tight capsule of ABS plastic.  Capsules are shown in the unassembled and assembled state.
}
\end{figure}





 \begin{figure}
\begin{overpic}[width = \columnwidth]{LotOneBouyancySim.pdf}\end{overpic}
\caption{
\label{fig:ContainerAndMRI}
(Left)  container vessel.  A molded plastic grid lines the container bottom.  (Right) experimental setup with the vessel and capsules inside a 600mm bore Siemens 3T MR-scanner.
}
\end{figure}
%%%%%%%%%%%%%%%
%%%%%%%%%%%%%%%
%%%%%%%%%%%%%%%%%%%%%%%%%%%%%%%%%%%%%%%%%%%%%%%%%%%%%%%%%%%
\section{Analysis}
\label{sec:analysis}
%%%%%%%%%%%%%%%%%%%%%%%%%%%%%%%%%%%%%%%%%%%%%%%%%%%%%%%%%%%


In mathematics, in the study of dynamical systems with two-dimensional phase space, a \emph{limit cycle} is a closed trajectory in phase space having the property that at least one other trajectory spirals into it either as time approaches infinity or as time approaches negative infinity. Such behavior is exhibited in some nonlinear systems. The study of limit cycles was initiated by Henri Poincar� (1854-1912.)
%%%%%%%%%%%%%%%
%%%%%%%%%%%%%%%
\input{experiment}
%%%%%%%%%%%%%%%
%%%%%%%%%%%%%%%
\input{conclusion}
%%%%%%%%%%%%%%%











    
    
\section{Acknowledgements}
We acknowledge Christos Bergelos and Panagiotis Vartholomeos for helpful discussion.
This work was supported by the National Science Foundation under
\href{http://nsf.gov/awardsearch/showAward?AWD_ID=1208509}{NRI-1208509}.  
   
\bibliographystyle{IEEEtran}
\bibliography{IEEEabrv,../bib/aaronrefs}%,../aaronrefs}
\end{document}


%/Users/ab55/Desktop/svn/MRSL-Papers/Drafts-Current/2013-03-13-IROS-MassiveUniformManipulation/document
%/Users/ab55/Desktop/svn/ensemble/bib





