%%%% Weekly Report Information %%%%
\newcommand{\handoutName}{Weekly report}
%\newcommand{\handoutdate}{August 25, 2015}  %use this to hard code the date
\newcommand{\handoutdate}{\today}
\newcommand{\duedate}{}


% Header template used for Weekly Reports
\documentclass[11pt,twoside]{article}

\setlength{\oddsidemargin}{0pt}
\setlength{\evensidemargin}{0pt}
\setlength{\textwidth}{6.5in}
\setlength{\topmargin}{0in}
\setlength{\textheight}{8.5in}
\setlength{\voffset}{0in}

\providecommand{\titlesize}{small}


\usepackage{graphicx}
\usepackage{subfigure}
\usepackage{palatino}
%\usepackage{cmbright}
\newcommand{\myMargin}{1.00in}
%\usepackage[pdftex]{hyperref}
\usepackage[small,bf]{caption}
\usepackage{amsmath}
\usepackage[usenames,dvipsnames]{color}
\usepackage{fancyhdr}
\pagestyle{fancy}
\usepackage{datetime}
\usepackage{fancyvrb}
\usepackage{color}
\usepackage[\titlesize, compact]{titlesec}
\usepackage{multicol}
\usepackage{enumitem}
\usepackage{pdfpages}
\usepackage{mdwlist}

\usepackage[colorlinks=true, urlcolor=blue, pdfborder={0 0 0}]{hyperref}

\newdateformat{dashdate}{\THEYEAR-\twodigit{\THEMONTH}-\twodigit{\THEDAY}}
\def\Tiny{\fontsize{3pt}{3pt}\selectfont}

\providecommand{\handoutName}{Handout title}
\providecommand{\handoutdate}{Handout date}
\providecommand{\duedate}{}

\lhead{Robotic Swarm Control Lab}
\chead{}
\rhead{ $<$LI HUANG$>$\\
\handoutdate }
\lfoot{}
\cfoot{\thepage}
\rfoot{\dashdate \Tiny \textcolor{Gray}{\today}}
\renewcommand{\headrulewidth}{0.4pt}
\renewcommand{\footrulewidth}{0.4pt}

\begin{document}

\vspace{0.60in}
\begin{center}
{\Large\textbf{\handoutName}}\\
\vspace{0.03in}
\textbf{\duedate}\\
\end{center}

\newcommand{\todo}[1]{
  \textcolor{Red}{
    \begin{tabular}{|c|}
      \hline
      \em \large \bfseries todo: \normalfont \normalsize #1 \\
      \hline
    \end{tabular}}
}

%%%%%%%%%%%%%%%%%%%%%%%%%%%%%%%%
%%
%%{\scriptsize
%%Weekly reports are to be emailed to atbecker@uh.edu by 5:00pm on Tuesdays.  The purpose of a weekly report is to:
%%(1) give you text and images for your papers, thesis, and dissertation, (2) document progress, (3) identify if you are stuck or need resources.
%%}
%%
%%%%%%%%%%%%%%%%%%%%%%%%%%%%%%%%
\section{My \emph{Goals} from last week}
\begin{itemize}
\item CASE 2016 video
\item Swarmathon algorithm



\end{itemize}
%%%%%%%%%%%%%%%%%%%%%%%%%%%%%%%%

\section{My \emph{Accomplishments} this week}


\begin{itemize}
\subsection{SMT video}
\item Uploaded the latest version video of SMT manifolds and amplifiers
\subsection{Swarmathon}
Finished Swarmathon search algorithm. The following items are the function I programmed. 
\item Each robot is assigned a unique ID via publisher to carry out different tasks.
\item The algorithm detects the boundary of the map to set up coordinate frame and grid division
\item The whole map is divided into 4 equal-size quadrants, and each of these 4 quadrants is further divided into 4 smaller equal-size, and so on till the smallest quadrant has a grid resolution $\le$ 0.8 m.
\item Assign weight to each grid.
\item As targets are detected in a grid, the grid weight increases and so do the neighboring grids
\item As a robot reaches a desired region but find targets already collected, decreases the grid weight
\item Each time, a robot is trying locate the largest weight grid and set for goal location.
\item The reason using grid division is to reduce on-board computation, so a robot does not have to scan the whole map to decide the next goal location.
   

%%
%%Deliverables can be documents, code, videos, plots, experiments, etc.	Save and backup all work [code, data, LaTeX, images, videos].  Papers and code should be saved to our git server: \href{github.com/aabecker}{https://github.com/aabecker}.
%%

%\item A couple of images are added here to show parts of my work.

\end{itemize}

%	\begin{figure}[h]
%		\begin{center}
%		\includegraphics[width = \columnwidth]{fig/schematic_R}
%
%		\vspace{1em}
%		\caption{}
%	\end{center}
%	\end{figure}




%%\section{Meeting with Dr. Becker on $<$INSERT DATE and TIME$>$}
%%
%%What I need Dr. Becker to do:
%%\begin{itemize}
%%\item something for Dr. Becker to do
%%\item $\ldots$
%%\end{itemize}
%%



%%
%%{\tiny
%%Template available at \href{https://github.com/aabecker/RoboticSwarmControlLab/blob/master/WeeklyReportTemplate/weekly_Report_Template.pdf}{https://github.com/aabecker/RoboticSwarmControlLab/blob/master/WeeklyReportTemplate/weekly\_Report\_Template.pdf}
%%}


\end{document}

